\documentclass[10pt,a4paper]{ctexart}
\usepackage[margin=2cm]{geometry}
\usepackage{graphicx}
\usepackage{hyperref}
\usepackage{amsmath}    % 提供方程环境
\usepackage{physics}    % 简化向量、微分算子语法
\newcommand{\myspace}[1]{\par\vspace{#1\baselineskip}} %自定义空一行
\usepackage{fancyhdr}
\usepackage{wrapfig}
\usepackage{booktabs} % 添加 booktabs 宏包
\pagestyle{fancy}
\fancyhf{}  % 清除所有页眉页脚
\fancyfoot[C]{\thepage}  % 在页脚中央设置页码
\renewcommand{\headrulewidth}{0pt}  % 去掉页眉横线
\renewcommand{\footrulewidth}{0pt}  % 去掉页脚横线

\usepackage{hyperref}
\hypersetup{
    colorlinks=true,
    linkcolor=blue,
    filecolor=magenta,      
    urlcolor=cyan,
    pdftitle={Overleaf Example},
    pdfpagemode=FullScreen,
    }

\usepackage{titlesec}

% 重新定义section格式
\titleformat{\section}
  {\normalfont\Large\bfseries}
  {\chinese{section}、}
  {0pt}
  {}

% 给出常用字号的自定义指令
\newcommand{\chuhao}{\fontsize{42pt}{48pt}\selectfont}
\newcommand{\xiaochu}{\fontsize{36pt}{42pt}\selectfont}
\newcommand{\xiaoer}{\fontsize{18pt}{21.6pt}\selectfont}
\newcommand{\sanhao}{\fontsize{16pt}{20pt}\selectfont}
\newcommand{\xiaosan}{\fontsize{15pt}{18pt}\selectfont}
\newcommand{\sihao}{\fontsize{14pt}{18pt}\selectfont}
\newcommand{\xiaosi}{\fontsize{12pt}{16pt}\selectfont}
\newcommand{\wuhao}{\fontsize{10.5pt}{12.6pt}\selectfont}

% 定义中文数字转换
\titleformat{\section}
  {\normalfont\Large\bfseries}
  {\zhnum{section}、}  % 使用\zhnum而不是\chinese
  {0pt}
  {}

% 标题设置
\title{\xiaochu\textbf{物理实验预习报告}}
\author{}
\date{}

\begin{document}

\vspace*{36pt} % 顶部空白

% 居中填写区域
\begin{center}
    \includegraphics[width = 9cm]{head.png}
    
    \xiaochu\textbf{物理实验预习报告}
    % 预留空白区域
    \vspace{13pt}
    \vspace{13pt}
    \vspace{13pt}
    \vspace{13pt}
    
    \sanhao\textbf{实验名称:\underline{\makebox[8cm]{ 光栅衍射实验 }}} \\[10pt]
    \sanhao\textbf{实验桌号:\underline{\makebox[8cm]{ 本次实验无 }}} \\[10pt]
    \sanhao\textbf{指导教师:\underline{\makebox[8cm]{ 王兆英 }}} \\[20pt]
    
    % 预留空白区域
    \myspace{7}
    
   \sihao\textbf{班级:\underline{\makebox[6cm]{机器人工程2402}}} \\[10pt]
    \sihao\textbf{姓名:\underline{\makebox[6cm]{毛挺}}} \\[10pt]
    \sihao\textbf{学号:\underline{\makebox[6cm]{3240104043}}} \\[30pt]
    
   \sihao\textbf{实验日期: \underline{\makebox[0.8cm]{2025}}年 \underline{\makebox[0.4cm]{11}}月\underline{\makebox[0.4cm]{18}}日  星期\underline{\makebox[0.6cm]{二}}下午}
    
    \vspace{18pt}
    \sihao 浙江大学物理实验教学中心
    
\end{center}

\newpage


    \section{实验综述}
    
    \xiaosi (自述实验现象、实验原理和实验方法,不超过500字,5分)
    
    \myspace{0}

    \begin{wrapfigure}{r}{0.4\textwidth}
    \centering
    \includegraphics[width=7cm]{透射光栅原理示意图.png}
    \caption{透射光栅原理示意图}
    \label{fig:principle}
    \end{wrapfigure}
    
    \
    
\subsubsection*{光栅衍射原理:}

    光栅由一组的等宽、等距和平行排列的狭缝组成,主要分透射光栅和反射光栅两种。本实验使用透射光栅,其原理如图1所示。其中,假设缝宽度为$b$,缝间距为$a$,则$d=a+b$为光栅的空间周期,也称为光栅常数。
    
    如图1所示,入射平行光垂直照射到透射光栅后发生衍射,当衍射角为时,相邻狭缝发出的衍射光光程差$\Delta = d \sin \theta $。根据光的相干条件,可推导得出光栅产生衍射亮条纹的条件
    \begin{align*}
        d\sin \theta = k \lambda  (k = \pm1, \pm2, \cdots, \pm n)
    \end{align*}

    其中,是$\theta $衍射角,是$ \lambda $光波长,$d$是光栅常数,$K$是光谱级数,上式也称为光栅方程。

    \subsubsection*{分光计测量光栅常数:}

    利用分光计测量透射光栅的光栅常数的原理如图2所示。将透射光栅放置于分光计载物台上,旋转载物台,使光栅平面垂直于平行光管。打开光源,转动分光计望远镜,即可观察光栅的各级衍射光谱。

    
    \begin{figure}[htbp]
    \centering
    \includegraphics[width=14cm]{使用分光计观察透射光栅衍射光谱示意图.png}
    \caption{使用分光计观察透射光栅衍射光谱示意图}
    \label{fig:principle}
    \end{figure}

    

    当入射光为复色光时,沿光栅法线方向的中央位置,能观察到白亮线,即$k=0$为级衍射线。
    在法线方向的两侧不同的角度方向上则可以观察到不同波长的光谱线,即为不同波长光的$\pm k$级衍射光谱,对称分布在中央零级光谱两侧。
    根据光栅方程$d \sin \theta  = k \lambda $,若光栅常数$d$已知,可以通过在实验中测得谱线的$\theta $和$k$,则可由光栅方程计算出谱线的波长$\lambda $;反之,如果已知谱线波长$\lambda $,也可求出光栅常数d。

    \subsubsection*{光栅的角色散率:}

    光栅的角色散率$D$描述了光栅作为色散元件将不同波长的光分开能力的大小,是衡量光栅分光能力的一个关键物理量。角色散率定义为同一级光谱中,单位波长间隔的光被分开的角度,即
    \begin{align*}
        D=\frac{\Delta \theta }{\Delta \lambda } 
    \end{align*}
    对于透射光栅,其角色散率可由光栅方程对两边求微分得出,即
    \begin{align*}
        D=\frac{ k }{d \cos \theta  } 
    \end{align*}

    由上式,光栅常数越小,光栅的角色散率越大。若光栅常数$d$已知,在实验中测得谱线的衍射级$k$次和衍射角$\theta $,则可计算出相应的光栅角色散率。

    \subsubsection*{分光计的调整:}

    按照 “分光计的调整与使用” 实验中的操作步骤,完成对分光计的调整。

    

    \subsubsection*{使用汞灯测量光栅常数$d$、角色散率$D$和不同颜色光波长:}
    
    光栅按图3所示置于载物台中心,其中a、b、c为载物台调节螺钉。调节载物台,使光栅面垂直于平行光管。转动望远镜筒,在光栅法线两侧观察各级衍射光谱。

    \begin{wrapfigure}{r}{0.4\textwidth}
    \centering
    \includegraphics[width=4cm]{透射光栅摆放位置示意图.png}
    \caption{透射光栅摆放位置示意图}
    \label{fig:principle}
    \end{wrapfigure}

    

    转动望远镜,使望远镜叉丝竖线与衍射光谱重合,测量$\pm 1$级绿色谱线的角坐标$ \varphi_1$和$\varphi _2$(为了消除偏心差,两侧游标都要读数的$ \varphi_1$和$\varphi _2$都要读),再测量$-1$级绿色谱线的角坐标$ \varphi_1'$和$\varphi _2'$。测量重复六次,列表统计,通过以下公式计算绿色谱线1级衍射的衍射角。
    \begin{align*}
        \theta = \frac{|\varphi_{1} - \varphi_{1}' | + |\varphi_{2} - \varphi_{2}' |}{4}
    \end{align*}

    \vspace{1cm}

    已知绿色谱线的波长为$546.7nm$,通过光栅方程$d \sin \theta  = k \lambda $计算光栅常数$d$及其不确定度,并通过$D=k/d \cos\theta $计算光栅绿色$\pm 1$级谱线的角色散率$D$及其不确定度。

    最后观察并测量紫、蓝、黄谱线的$\pm 1$级衍射光的衍射角,列表统计。根据测得的光栅常数$d$和衍射角$\theta $,计算紫、蓝、黄谱线的波长。
   
    
    \section{实验重点}

    \xiaosi(简述本实验的学习重点,不超过100字,3分)


    \begin{enumerate}   
        \item 掌握分光计的调节和使用方法;
        \item 理解光栅衍射的基本原理;
        \item 使用分光计测量光栅常数与光栅角分辨率;
        \item 使用光栅测量汞灯谱线波长。

    \end{enumerate}  
    
    
    \section{实验难点}
    
    \xiaosi(简述本实验的实现难点,不超过100字,2分)

    \begin{enumerate}    
        \item 分光计调节难:需依次粗调水平、调望远镜焦至无穷远、用逐次减半法调光轴与主轴垂直,步骤多且精度要求高;
        \item 光栅定位难:需精准放置于载物台中心,确保光栅面垂直平行光管,偏差会影响衍射光谱观察;
        \item 衍射角测量难:需读两侧游标消除偏心差,重复测量 6 次,操作繁琐且需把控读数精度;
        \item 二级光谱观测难:光强弱,需调节狭缝或环境光,分辨与测量难度高于一级光谱。
    \end{enumerate}  \



\newpage

\sihao\textbf{注意事项:}
\xiaosi
\begin{enumerate}
    \item  用 PDF 格式上传“实验报告”,文件名:学生姓名+学号+实验名称+周次。
    \item  “实验报告”必须递交在“学在浙大”的本课程的对应实验项目的“作业”模块内。
    \item  “实验报告”成绩必须在“浙江大学物理实验教学中心网站”-“选课系统”内查询。
    \item 教学评价必须在“浙江大学物理实验教学中心网站”-“选课系统”内进行,学生必须进行教学评价,才能看到实验报告成绩,教学评价必须在本次实验结束后 3 天内进行。
\end{enumerate}

\vspace{1cm}
\xiaosi\centering \textbf{浙江大学物理实验教学中心制}

\end{document}