\documentclass[10pt,a4paper]{ctexart}
\usepackage[margin=2cm]{geometry}
\usepackage{graphicx}
\usepackage{hyperref}
\usepackage{amsmath}    % 提供方程环境
\usepackage{physics}    % 简化向量、微分算子语法
\newcommand{\myspace}[1]{\par\vspace{#1\baselineskip}} %自定义空一行
\usepackage{fancyhdr}
\usepackage{wrapfig}
\usepackage{multirow}
\usepackage{booktabs} % 添加 booktabs 宏包
\pagestyle{fancy}
\fancyhf{}  % 清除所有页眉页脚
\fancyfoot[C]{\thepage}  % 在页脚中央设置页码
\renewcommand{\headrulewidth}{0pt}  % 去掉页眉横线
\renewcommand{\footrulewidth}{0pt}  % 去掉页脚横线
% hyperref 已在上方加载一次,避免重复加载
\hypersetup{
    colorlinks=true,
    linkcolor=blue,
    filecolor=magenta,      
    urlcolor=cyan,
    pdftitle={Overleaf Example},
    pdfpagemode=FullScreen,
    }

\usepackage{titlesec}

% 重新定义section格式
\titleformat{\section}
  {\normalfont\Large\bfseries}
  {\chinese{section}、}
  {0pt}
  {}

% 给出常用字号的自定义指令
\newcommand{\chuhao}{\fontsize{42pt}{48pt}\selectfont}
\newcommand{\xiaochu}{\fontsize{36pt}{42pt}\selectfont}
\newcommand{\xiaoer}{\fontsize{18pt}{21.6pt}\selectfont}
\newcommand{\sanhao}{\fontsize{16pt}{20pt}\selectfont}
\newcommand{\xiaosan}{\fontsize{15pt}{18pt}\selectfont}
\newcommand{\sihao}{\fontsize{14pt}{18pt}\selectfont}
\newcommand{\xiaosi}{\fontsize{12pt}{16pt}\selectfont}
\newcommand{\wuhao}{\fontsize{ 10.5pt}{12.6pt}\selectfont}

% 定义中文数字转换
\titleformat{\section}
  {\normalfont\Large\bfseries}
  {\zhnum{section}、}  % 使用\zhnum而不是\chinese
  {0pt}
  {}

% 标题设置
\title{\xiaochu\textbf{物理实验报告}}
\author{}
\date{}

\begin{document}

\vspace*{36pt} % 顶部空白

% 居中填写区域
\begin{center}
    \includegraphics[width = 9cm]{figures/head.png}
    
    \xiaochu\textbf{物理实验报告}
    % 预留空白区域
    \vspace{13pt}
    \vspace{13pt}
    \vspace{13pt}
    \vspace{13pt}
    
    \sanhao\textbf{实验名称:\underline{\makebox[8cm]{ 示波器的调整和使用 }}} \\[10pt]
    \sanhao\textbf{实验桌号:\underline{\makebox[8cm]{ }}} \\[10pt]
    \sanhao\textbf{指导教师:\underline{\makebox[8cm]{  }}} \\[20pt]
    
    % 预留空白区域
    \myspace{7}
    
    \sihao\textbf{班级:\underline{\makebox[6cm]{cc98}}} \\[10pt]
    \sihao\textbf{姓名:\underline{\makebox[6cm]{Hydrofoil}}} \\[10pt]
    \sihao\textbf{学号:\underline{\makebox[6cm]{324010}}} \\[30pt]
    
    \sihao\textbf{实验日期: \underline{\makebox[0.8cm]{2025}}年 \underline{\makebox[0.4cm]{10}}月\underline{\makebox[0.4cm]{29}}日  星期\underline{\makebox[0.6cm]{四}}上午}
    
    \vspace{18pt}
    \sihao 浙江大学物理实验教学中心
    
\end{center}

\newpage

\section{预习报告}


    \subsection{实验综述}  

    \xiaosi (自述实验现象、实验原理和实验方法,不超过500字,5分)


    \begin{wrapfigure}{r}{0.4\textwidth}
    \centering
    \includegraphics[width=8cm]{figures/分光计.png}
    \caption{分光计原理}
    \label{fig:principle}
    \end{wrapfigure}

    \

    \subsubsection*{反射法测量三棱镜棱角原理:}
    三棱镜中相邻两个光学平面之间的夹角称为棱角(顶角)。用一束平行光射到三棱镜的棱角(如右图所示),光线 1 经 AB 面反射,光线 2 经 AC 面反射,两反射光线的夹角为$ \alpha $,两反射光线的夹角$\alpha $与棱角 A 的关系可由几何光学求得:

    设两读数窗为I窗和II窗,当望远镜在右边时,读得两窗读数为: $ \angle \text{右}_I$   和 $ \angle \text{右}_{II} $ ;
    同理,当望远镜在左边时,读得两窗读数为: $ \angle \text{左}_{I} $ 和 $ \angle \text{左}_{II} $ 。
    则: $ \alpha_I = \angle \text{右}_{I} - \angle \text{左}_{I} $ ,$ \alpha_{II} = \angle \text{右}_{II} - \angle \text{左}_{II} $。为了消除仪器的偏心差,取 $ \varphi  = \frac{\varphi_{1}+\varphi_{2}}{2} $ ,
    故棱角 $ \angle A$ 的计算公式:$\angle A = \frac{| \angle \text{右}_{I} - \angle \text{左}_{I} | + | \angle \text{右}_{II} - \angle \text{左}_{II} |}{4} $ 。

    \subsubsection*{自准直法测量原理:}

    在载物平台上放一镜面垂直于望远镜光轴的平面反射镜,调节亮十字与物镜之间的距离(即调焦)。如果亮十字恰好处于物镜的焦平面上,那么亮十字上任意一点发出的光经物镜变为平行光,此平行光由反射镜反射回来,经物镜后所成亮十字像应准确地处在亮十字所在平面上。因此在调焦过程中,只要在亮十字所在平面上看到反射回来的清晰亮十字像时,望远镜已调焦至无穷远,这种调焦方法称为自准直法,光路如下图所示:

     \begin{figure}[htbp]
        \centering
        \includegraphics[width=8cm]{figures/光路图.png}
        \caption{调焦光路图}
    \label{fig:optics}
    \end{figure}


    \subsubsection*{分光计的调整:}

    保证入射光线是平行光,望远镜能接收平行光,平行光管和望远镜的光轴与分光计中心轴垂直。

    \textbf{(1)粗调:} 用目测法调节望远镜倾斜度调节螺钉,使望远镜光轴基本与分光计中心轴垂直。

    \textbf{(2)望远镜调焦至无穷远:}
        
    平面镜置于载物平台;

    调节目镜直至看清 “丰” 形叉丝;

    调节望远镜倾斜螺钉直至找到亮十字像为止;
        
    调节调焦螺钉直至看见清晰亮十字,并使亮十字像与 “丰” 形叉丝上刻线重合。

    \textbf{(3)调整望远镜光轴、载物平台面分别与分光计中心转轴垂直:}

    调节载物台三只倾斜度调节螺钉中的两只,使反光镜两面反射的亮十字像重合于 “丰” 形叉丝的上刻线;

    将反光镜置于与载物台某两脚连线平行的平台面直径上,调节第三只螺钉,使亮十字像与 “丰” 形叉丝上刻线重合。

   \textbf{(4)调整平行光管光轴与分光计中心轴垂直。}

    
    \subsubsection*{测量三棱镜棱角:}
    
    将三棱镜置于载物台,顶角对准平行光管的中心,且顶角应接近平台中心略偏上的位置,测量左右两反射光线的角位置,每次测量稍改变顶角接近平台中心的位置。

    
    \subsection{实验重点}

    \xiaosi(简述本实验的学习重点,不超过100字,3分)


    \begin{enumerate}   
        \item 了解分光计的结构;
        \item 学会正确的分光计调节和使用方法;
        \item 利用分光计测量三棱镜的底角(顶角)。

    \end{enumerate}  
    
    
    \subsection{实验难点}
    
    \xiaosi(简述本实验的实现难点,不超过100字,2分)

    \begin{enumerate}    
        \item 若目测望远镜或载物平台明显不水平,在望远镜中将难以找到绿色十字像;
        \item 实验时需要对望远镜进行自准调焦,使得绿色十字像处于最清晰状态;
        \item 实验过程中,望远镜和载物平台调整好后,它们的倾斜调节螺钉都不可再随意转动;
        \item 三棱镜的顶角应接近平台中心偏上的位置,否则在望远镜中将难以看到反射光;
    \end{enumerate}  


\section{原始数据}

    \xiaosi (将有老师签名的“自备数据记录草稿纸”的扫描或手机拍摄图粘贴在下方,20分)

    \begin{figure}[htbp]
        \centering
        \includegraphics[width=16cm]{figures/实验数据.jpg}
        \caption{实验数据}
        \label{fig:图2}
    \end{figure}

    
\section{结果与分析}

    \subsection{数据处理与结果}
    \xiaosi (列出数据表格、选择数据处理方法、给定测量或计算结果,30分)

        % 注意:编译此表格需要您的文档已包含 \usepackage{ctex} (或 \usepackage[UTF8]{ctex}) 和 \usepackage{amsmath}
        \begin{table}[htbp]
            \centering
            \caption{三棱镜测量数据}
            \label{tab:bridge_custom}
            \begin{tabular}{cccccccc}
                \toprule
                \multirow{2}{*}{实验次数} & \multicolumn{2}{c}{左} & \multicolumn{2}{c}{右} & \multirow{2}{*}{$\varphi_I = |\theta_{\text{左}I} - \theta_{\text{右}I}|$} &  \multirow{2}{*}{$\varphi_{II} = |\theta_{\text{左}II} - \theta_{\text{右}II}|$} & \multirow{2}{*}{$\varphi = \frac{1}{2}(\varphi_I + \varphi_{II})$} \\
                \cmidrule(lr){2-3} \cmidrule(lr){4-5}

                & \multicolumn{1}{c|}{$\Theta_{\text{左}I}$} & $\Theta_{\text{左}II}$ & \multicolumn{1}{c|}{$\Theta_{\text{右}I}$} & $\Theta_{\text{右}II}$ & & & \\ 

                 \midrule
                1 & $319^\circ 16'$ & $138^\circ 15'$ & $199^\circ 32'$ & $18^\circ 20'$ & $119^\circ 44'$ & $119^\circ 55'$ & $119^\circ 50'$ \\
                2 & $311^\circ 43'$ & $131^\circ 59'$ & $192^\circ 1'$ & $12^\circ 2'$ & $119^\circ 42'$ & $119^\circ 57'$ & $119^\circ 50'$ \\            
                3 & $317^\circ 10'$ & $137^\circ 26'$ & $197^\circ 24'$ & $17^\circ 30'$ & $119^\circ 46'$ & $119^\circ 56'$ & $119^\circ 51'$ \\
                4 & $296^\circ 48'$ & $96^\circ 47'$ & $177^\circ 3'$ & $336^\circ 49'$ & $119^\circ 45'$ & $119^\circ 58'$ & $119^\circ 52'$ \\
                5 & $253^\circ 48'$ & $73^\circ 51'$ & $133^\circ 51'$ & $313^\circ 53'$ & $119^\circ 57'$ & $119^\circ 58'$ & $119^\circ 58'$ \\
                6 & $274^\circ 32'$ & $94^\circ 29'$ & $154^\circ 33'$ & $334^\circ 43'$ & $119^\circ 59'$ & $119^\circ 46'$ & $119^\circ 53'$ \\
                \bottomrule
            \end{tabular}
        \end{table}

        而 $ \angle A = \frac{1}{2} \varphi $, 得到下表:

        \begin{table}[htbp]
            \centering
            \caption{三棱镜测量数据}
            \label{tab:bridge_custom}
            \begin{tabular}{cc}
                \toprule
                实验次数 & $\angle A$ \\
                \midrule
                1 & $59^\circ 55'$ \\
                2 & $59^\circ 55'$ \\
                3 & $59^\circ 56'$ \\
                4 & $59^\circ 56'$ \\
                5 & $59^\circ 59'$ \\
                6 & $59^\circ 57'$ \\
                \bottomrule
            \end{tabular}
        \end{table}

        进一步可得:
        \begin{align*}
        \text{平均棱角:}\overline{A} &= \frac{1}{6}\sum_{i=1}^{6}A_{i}=59^{\circ}56' \\
        \text{不确定度计算:}U_A &= \sqrt{\frac{1}{6\times(6 - 1)}\sum_{i=1}^{6}(\angle A_{i}-\angle \overline{A})^{2}} = 1'\\
        U_B &= \frac{\Delta_{\text{仪}}}{\sqrt{3}} = 1' \\
        \text{即:}u &= \sqrt{U_A^{2}+U_B^{2}} = 1' \\
        \angle A &= 59^{\circ}56'\pm1' \\
        \end{align*}

        由此,$ \angle A $ 最终测量结果:\(\angle A=59^{\circ}56'\pm1'\)

    \subsection{误差分析}
    \xiaosi (运用测量误差、相对误差、不确定度等分析实验结果,20分)
    
    \subsubsection*{(一)系统误差}
    
    仪器老化、刻度不准等因素造成\(\Delta_{\text{仪}}= \pm 1'\);同时,环境温度引起仪器热胀冷缩,进而影响测量结果。
    
    (这些误差都较小,本次实验测量结果较为准确,不确定度仅为 1′。)
    
    \subsubsection*{(二)偶然误差}

    \begin{enumerate}
        \item 载物台难以完全调平,反射光与目镜中竖直轴线存在一定的夹角,导致光路出现一定偏差,影响最终测量结果。
        \item 反射光有时较弱,与周围环境光线区分度不大,使得其与竖直轴线对齐的操作难度增加,产生误差。
        \item 对刻度盘上的刻度进行读数时,判断刻度线是否对齐存在主观性,会有一定误差。
        \item 在微调过程中,绿色亮十字与刻度线对齐操作判断存在主观性,难以完全对齐,导致望远镜光轴、载物台平面不能与分光计中心轴完全垂直。
    \end{enumerate}
    
    
    \subsection{实验探讨}
    \xiaosi (对实验内容、现象和过程的小结,不超过100字,10分)

    \myspace{1}

本次实验学习了分光计的使用方法,并对三棱镜的顶角进行了细致、较为精确的测量。实验原理简单明晰,但实验操作步骤较多、测量数据量较大、操作精度要求较高,故本次实验很好地锻炼了我的动手能力、观察能力和将实验理论转化为实际操作的应用能力,培养了我的耐心。同时,不确定度计算等内容也增强了我的误差分析能力、数据处理能力,为后续的系列实验打下了良好的基础。
\section{思考题}

    \xiaosi (解答教材或讲义或老师布置的思考题,10分)

    \subsubsection*{测量三棱镜棱角时,棱镜摆放的位置该怎么选,有区别吗?}

    答:三棱镜顶角为什么应接近平台中心偏上一点点位置。因为实验中,希望测量的角度是反射光线之间的夹角,而测量刻度以分光计的中心作为顶点,延长光路后可发现三棱镜顶角与光路交点不重合,如此放置可减小这一误差。
        
    \subsubsection*{为什么狭缝要调节至适当宽度(1-2mm)?太宽、太窄有什么问题?}
    
    答:狭缝调至 1-2mm 是为让平行光管出射合适平行光,保证测量精度。太宽时,出射光非严格平行,会使谱线展宽模糊,导致角度测量误差大;太窄则透光量不足,光线过弱,难以清晰观测到反射光或谱线,影响实验正常进行。
        
    \subsubsection*{粗调时,为什么会出现一面有十字像,转了180°没有十字像?这时该如何调节,请简要描述?}
    
    答:因载物台或望远镜倾斜,反光镜两面反射光未进入望远镜,故一面有像、转180°无像。调节方法:先调望远镜倾斜螺钉,降低有像一侧高度或升高另一侧;再微调载物台对应螺钉,使反光镜两面反射光均能进入望远镜,直至转180°后也能看到十字像。
       
    \subsubsection*{你可以用别的方法测量三棱镜顶角吗?}

    答:可以用自准直法测三棱镜顶角。将三棱镜放在载物台,使顶角对着望远镜,调节望远镜,分别让其光轴与三棱镜两光学面垂直,从读数装置读得两次望远镜方位角,两角度差值的补角即为顶角。该方法无需平行光入射,操作相对简便,能避免反射法中平行光对准的误差。
    
    

\newpage

\sihao\textbf{注意事项:}
\xiaosi
\begin{enumerate}
    \item  用 PDF 格式上传“实验报告”,文件名:学生姓名+学号+实验名称+周次。
    \item  “实验报告”必须递交在“学在浙大”的本课程的对应实验项目的“作业”模块内。
    \item  “实验报告”成绩必须在“浙江大学物理实验教学中心网站”-“选课系统”内查询。
    \item 教学评价必须在“浙江大学物理实验教学中心网站”-“选课系统”内进行,学生必须进行教学评价,才能看到实验报告成绩,教学评价必须在本次实验结束后 3 天内进行。
\end{enumerate}

\vspace{1cm}
\xiaosi\centering \textbf{浙江大学物理实验教学中心制}

\end{document}