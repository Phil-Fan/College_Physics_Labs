\documentclass[10pt,a4paper]{ctexart}
\usepackage[margin=2cm]{geometry}
\usepackage{graphicx}
\usepackage{hyperref}
\usepackage{amsmath}    % 提供方程环境
\usepackage{physics}    % 简化向量、微分算子语法
\newcommand{\myspace}[1]{\par\vspace{#1\baselineskip}} %自定义空一行
\usepackage{fancyhdr}
\usepackage{wrapfig}
\usepackage{booktabs} % 添加 booktabs 宏包
\pagestyle{fancy}
\fancyhf{}  % 清除所有页眉页脚
\fancyfoot[C]{\thepage}  % 在页脚中央设置页码
\renewcommand{\headrulewidth}{0pt}  % 去掉页眉横线
\renewcommand{\footrulewidth}{0pt}  % 去掉页脚横线

\usepackage{hyperref}
\hypersetup{
    colorlinks=true,
    linkcolor=blue,
    filecolor=magenta,      
    urlcolor=cyan,
    pdftitle={Overleaf Example},
    pdfpagemode=FullScreen,
    }

\usepackage{titlesec}

% 重新定义section格式
\titleformat{\section}
  {\normalfont\Large\bfseries}
  {\chinese{section}、}
  {0pt}
  {}

% 给出常用字号的自定义指令
\newcommand{\chuhao}{\fontsize{42pt}{48pt}\selectfont}
\newcommand{\xiaochu}{\fontsize{36pt}{42pt}\selectfont}
\newcommand{\xiaoer}{\fontsize{18pt}{21.6pt}\selectfont}
\newcommand{\sanhao}{\fontsize{16pt}{20pt}\selectfont}
\newcommand{\xiaosan}{\fontsize{15pt}{18pt}\selectfont}
\newcommand{\sihao}{\fontsize{14pt}{18pt}\selectfont}
\newcommand{\xiaosi}{\fontsize{12pt}{16pt}\selectfont}
\newcommand{\wuhao}{\fontsize{10.5pt}{12.6pt}\selectfont}

% 定义中文数字转换
\titleformat{\section}
  {\normalfont\Large\bfseries}
  {\zhnum{section}、}  % 使用\zhnum而不是\chinese
  {0pt}
  {}

% 标题设置
\title{\xiaochu\textbf{物理实验预习报告}}
\author{}
\date{}

\begin{document}

\vspace*{36pt} % 顶部空白

% 居中填写区域
\begin{center}
    \includegraphics[width = 9cm]{head.png}
    
    \xiaochu\textbf{物理实验预习报告}
    % 预留空白区域
    \vspace{13pt}
    \vspace{13pt}
    \vspace{13pt}
    \vspace{13pt}
    
    \sanhao\textbf{实验名称:\underline{\makebox[8cm]{ 双臂电桥测低电阻 }}} \\[10pt]
    \sanhao\textbf{实验桌号:\underline{\makebox[8cm]{  }}} \\[10pt]
    \sanhao\textbf{指导教师:\underline{\makebox[8cm]{  }}} \\[20pt]
    
    % 预留空白区域
    \myspace{7}
    
  \sihao\textbf{班级:\underline{\makebox[6cm]{cc98}}} \\[10pt]
    \sihao\textbf{姓名:\underline{\makebox[6cm]{Hydrofoil}}} \\[10pt]
    \sihao\textbf{学号:\underline{\makebox[6cm]{324010}}} \\[30pt]
    
   \sihao\textbf{实验日期: \underline{\makebox[0.8cm] 2025 }年 \underline{\makebox[0.4cm]10}月\underline{\makebox[0.4cm]15}日  星期\underline{\makebox[0.6cm]三}上午}
    
    \vspace{18pt}
    \sihao 浙江大学物理实验教学中心
    
\end{center}

\newpage


    \section{实验综述}
    
    \xiaosi (自述实验现象、实验原理和实验方法,不超过500字,5分)
    
    \myspace{0}

    
    \begin{wrapfigure}{r}{0.4\textwidth}
    \centering
    \includegraphics[width=7cm]{双臂电桥2.png}
    \includegraphics[width=7cm]{双臂电桥.png}
    \caption{双臂电桥电路图}
    \label{fig:图1}
    \end{wrapfigure}
    
    \
    
    \subsubsection*{双臂电桥测量原理:}

    右图是一个完整的低值电阻,其中\(C_1'\)和\(C_2'\)均为电流接头,\(P_1\)、\(P_2\)为电位接头,电位接头间的电阻才是实测电阻\(R_x\)。    

    将采用四端接入法的低电阻(如带测小电阻和比较臂低电阻)接入原单臂电桥,等效电阻图如右下图2所示。

    为了消除(或减小)附加电阻(如接线电阻和导线电阻)的影响,分别接入了阻值均大于 10Ω 的标准电阻\(R_3\)、\(R_4\),且为考虑平衡时$\frac{R_1}{R_2}$和$\frac{R_3}{R_4}$的差别对测量结果的影响,用阻值小于$0.001\Omega$的粗导线r来连接电阻\(R_x\)和\(R_s\)。此外,电路中加接一放大器用以增加灵敏度,使不平衡电流Ig经过放大再由检流计指示。

    电桥平衡时,Ig=0,可得以下公式:
    \begin{gather*}
    I_{1} R_{1}=I_{3} R_{x}+I_{2} R_{3} \\
    I_{1} R_{2}=I_{3} R_{4}+I_{2} R_{3} \\
    I_{2}\left(R_{3}+R_{4}\right)=\left(I_{3}-I_{2}\right) r
    \end{gather*}
    
    进而推导得出:
    \begin{equation*}
        R_{x}=\frac{R_{1}}{R_{2}} R_{4}+\frac{R_{3} \cdot r}{R_{3}+R_{4}+r}\left(\frac{R_{1}}{R_{2}}-\frac{R_{3}}{R_{4}}\right)
    \end{equation*}
    
    当\(\frac{R_{1}}{R_{2}}=\frac{R_{3}}{R_{4}}\)时,有:
    \begin{equation*}
    R_{x}=\frac{R_{1}}{R_{2}} R_{4}
    \end{equation*}

    测量金属导体电阻温度系数时,存在以下关系:
    \begin{equation*}
    R=R_{0}\left(1+\alpha t+\beta t^{2}+\gamma t^{3}+\cdots\right)
    \end{equation*}   

    式中,式中,R是\(t^{\circ}C\)时的阻值,\(R_0\)是\(0^{\circ}C\)时的阻值,\(\alpha\)、\(\beta\)、\(\gamma\)是该材料的电阻温度系数。温度不太高时,上式可近似为:
    \begin{equation*}
    R=R_{0}(1+\alpha t)
    \end{equation*}

    为避免\(R_0\)的测量,可设\(R_{x1}=R(t_1)\)、\(R_{x2}=R(t_2)\),则:
    \begin{equation*}
    \alpha=\frac{R_{x2}-R_{x1}}{R_{x1} t_2 - R_{x2} t_1}
    \end{equation*}

    \subsubsection*{测量金属导体的电阻率:}

    \begin{enumerate}
        \item 将待测导体接入电位接头(\(P_1\)、\(P_2\))和电流接头(\(C_1'\)、\(C_2'\)),按步骤将“电阻粗调”示数加上“电阻细调”示数乘上倍率,读得阻值R。再用游标卡尺测出待测金属导体的直径d,读取电位接头间的长度l。由以上数据,可算出导体的电阻率$\rho =R\cdot \frac{S}{l} = R\cdot \frac{\pi d^2}{4l} $;
        \item 分别计算出R,d,l的不确定度,计算相对不确定度\(\frac{U(\rho)}{\rho}\);
        \item 得出电阻率的结果表达式$\rho = \overline{\rho} \pm U(\rho) $。
    \end{enumerate}

    \subsubsection*{测量金属导体的电阻温度系数:}
    
    \begin{enumerate}
        \item 将待测电阻封装在加热炉内并浸泡在机油中,采用升温法完成后续测量;
        \item \textbf{升温法:}将温控仪电源开启,显示屏显示当前温度。将 “转换开关” 设置于 “设定” 档,按动调节按钮,根据实验需要设定加热温度上限,再将转换开关置于 “运行” 档位。加热过程中,调节双臂电桥进行低电阻测量,每隔 5℃左右记录一次阻值及其对应的温度值。
        \item \textbf{降温法:}先将待测电阻加热至一定温度,然后开启风扇降温,在降温过程中,约每 5℃记录一次电阻值对应的阻值及温度值。
        \item 整理实验数据,将数据分别代入\(R=R_0(1+\alpha t)\)中,计算出平均温度系数\(\overline{\alpha}\);
        \item 绘制\(R-t\)特性曲线,根据曲线求温度系数\(\alpha\),并计算相对误差。
    \end{enumerate}


    \section{实验重点}
    
    \xiaosi(简述本实验的学习重点,不超过100字,3分)

    \begin{enumerate}
        \item 掌握双臂电桥测低电阻的原理和使用方法;
        \item 了解单臂电桥与双臂电桥的联系与区别;
        \item 掌握 Q-44 型电桥测量低电阻的操作方法。
    \end{enumerate}
    
    \section{实验难点}
    
    \xiaosi(简述本实验的实现难点,不超过100字,2分)

    \begin{enumerate}
        \item 实验中电阻温度变化快,读取数据应在同一时刻,否则会造成较大误差;
        \item 应加热装置 PID 调节需反复微调,响应滞后,易超调或不达设定温,影响数据稳定性,难满足实验精度。
    \end{enumerate}



\newpage

\sihao\textbf{注意事项:}
\xiaosi
\begin{enumerate}
    \item  用 PDF 格式上传“实验报告”,文件名:学生姓名+学号+实验名称+周次。
    \item  “实验报告”必须递交在“学在浙大”的本课程的对应实验项目的“作业”模块内。
    \item  “实验报告”成绩必须在“浙江大学物理实验教学中心网站”-“选课系统”内查询。
    \item 教学评价必须在“浙江大学物理实验教学中心网站”-“选课系统”内进行,学生必须进行教学评价,才能看到实验报告成绩,教学评价必须在本次实验结束后 3 天内进行。
\end{enumerate}

\vspace{1cm}
\xiaosi\centering \textbf{浙江大学物理实验教学中心制}

\end{document}