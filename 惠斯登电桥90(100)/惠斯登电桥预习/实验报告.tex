\documentclass[10pt,a4paper]{ctexart}
\usepackage[margin=2cm]{geometry}
\usepackage{graphicx}
\usepackage{hyperref}
\usepackage{amsmath}    % 提供方程环境
\usepackage{physics}    % 简化向量、微分算子语法
\newcommand{\myspace}[1]{\par\vspace{#1\baselineskip}} %自定义空一行
\usepackage{fancyhdr}
\usepackage{wrapfig}
\usepackage{booktabs} % 添加 booktabs 宏包
\pagestyle{fancy}
\fancyhf{}  % 清除所有页眉页脚
\fancyfoot[C]{\thepage}  % 在页脚中央设置页码
\renewcommand{\headrulewidth}{0pt}  % 去掉页眉横线
\renewcommand{\footrulewidth}{0pt}  % 去掉页脚横线

\usepackage{hyperref}
\hypersetup{
    colorlinks=true,
    linkcolor=blue,
    filecolor=magenta,      
    urlcolor=cyan,
    pdftitle={Overleaf Example},
    pdfpagemode=FullScreen,
    }

\usepackage{titlesec}

% 重新定义section格式
\titleformat{\section}
  {\normalfont\Large\bfseries}
  {\chinese{section}、}
  {0pt}
  {}

% 给出常用字号的自定义指令
\newcommand{\chuhao}{\fontsize{42pt}{48pt}\selectfont}
\newcommand{\xiaochu}{\fontsize{36pt}{42pt}\selectfont}
\newcommand{\xiaoer}{\fontsize{18pt}{21.6pt}\selectfont}
\newcommand{\sanhao}{\fontsize{16pt}{20pt}\selectfont}
\newcommand{\xiaosan}{\fontsize{15pt}{18pt}\selectfont}
\newcommand{\sihao}{\fontsize{14pt}{18pt}\selectfont}
\newcommand{\xiaosi}{\fontsize{12pt}{16pt}\selectfont}
\newcommand{\wuhao}{\fontsize{10.5pt}{12.6pt}\selectfont}

% 定义中文数字转换
\titleformat{\section}
  {\normalfont\Large\bfseries}
  {\zhnum{section}、}  % 使用\zhnum而不是\chinese
  {0pt}
  {}

% 标题设置
\title{\xiaochu\textbf{物理实验预习报告}}
\author{}
\date{}

\begin{document}

\vspace*{36pt} % 顶部空白

% 居中填写区域
\begin{center}
    \includegraphics[width = 9cm]{figures/head.png}
    
    \xiaochu\textbf{物理实验预习报告}
    % 预留空白区域
    \vspace{13pt}
    \vspace{13pt}
    \vspace{13pt}
    \vspace{13pt}
    
    \sanhao\textbf{实验名称:\underline{\makebox[8cm]{ 惠斯登电桥 }}} \\[10pt]
    \sanhao\textbf{实验桌号:\underline{\makebox[8cm]{ }}} \\[10pt]
    \sanhao\textbf{指导教师:\underline{\makebox[8cm]{ }}} \\[20pt]
    
    % 预留空白区域
    \myspace{7}
    
   \sihao\textbf{班级:\underline{\makebox[6cm]{cc98}}} \\[10pt]
    \sihao\textbf{姓名:\underline{\makebox[6cm]{Hydrofoil}}} \\[10pt]
    \sihao\textbf{学号:\underline{\makebox[6cm]{324010}}} \\[30pt]
    
   \sihao\textbf{实验日期: \underline{\makebox[0.8cm] 2025 }年 \underline{\makebox[0.4cm]9}月\underline{\makebox[0.4cm]25}日  星期\underline{\makebox[0.6cm]四}上午}
    
    \vspace{18pt}
    \sihao 浙江大学物理实验教学中心
    
\end{center}

\newpage



    \section{实验综述}
    
    \xiaosi (自述实验现象、实验原理和实验方法,不超过500字,5分)
    
    \myspace{0}

    \begin{wrapfigure}{r}{0.4\textwidth}
    \centering
    \includegraphics[width=7cm]{figures/电桥.png}
    \caption{惠斯通电桥电路图}
    \label{fig:图1}
    \end{wrapfigure}

    \
    
    \subsubsection*{惠斯登电桥测量电阻的原理:}
    右图为惠斯登电桥的原理图,\(R_{1}\)、\(R_{2}\)、\(R_{s}\)、\(R_{x}\)组成 “桥臂”,G 和 S 组成 “桥路”。当通过检流计 G 的电流\(I_{g}\)为 0 时,B、D 电位相同,电桥达到平衡,此时流过电阻\(R_{1}\)和\(R_{x}\)的电流同为\(I_{1}\),流过电阻\(R_{2}\)和\(R_{s}\)的电流同为\(I_{2}\),即\(U_{AB}=I_{1}R_{1}=U_{AD}=I_{2}R_{2}\),因此有\(\frac{I_{1}}{I_{2}}=\frac{R_{2}}{R_{1}}\),即\(R_{x}=\frac{R_{1}}{R_{2}}R_{s}\)。该式即为电桥的平衡条件,只要调节\(R_{s}\)使检流计无电流通过,并记下\(R_{s}\)数值,就能得出\(R_{x}\)。
    
    \subsubsection*{交换法减小自组电桥系统误差:}
    在上图所示自组电桥中,若电桥的灵敏度足够高,则系统误差主要由\(R_{1}\)、\(R_{2}\)、\(R_{s}\)自身的误差决定,此时相对不确定度为\(\frac{\Delta R_{x}}{R_{x}}=\sqrt{(\frac{\Delta R_{1}}{R_{1}})^{2}+(\frac{\Delta R_{2}}{R_{2}})^{2}+(\frac{\Delta R_{s}}{R_{s}})^{2}}\)。为尽量减小系统误差,可在电桥调节平衡后,互换\(R_{x}\)和\(R_{s}\),设\(R_{s}\)变为\(R_{s}'\)时电桥重新达到平衡,这时有\(R_{x}=\frac{R_{2}}{R_{1}}R_{s}'\),又因\(R_{x}=\frac{R_{1}}{R_{2}}R_{s}\),故有\(R_{x}=\sqrt{R_{s}R_{s}'}\),这样就消除了\(R_{1}\)、\(R_{2}\)自身误差对测量误差的影响。如此一来,\(R_{x}\)的不确定度为\(\frac{\Delta R_{x}}{R_{x}}=\frac{1}{2}\sqrt{(\frac{\Delta R_{s}}{R_{s}})^{2}+(\frac{\Delta R_{s}'}{R_{s}'})^{2}}\approx\frac{\Delta R_{s}}{R_{s}}\),其只与\(R_{s}\)的仪器误差有关,而\(R_{s}\)可选用具有一定精度的标准电阻箱,以此减小\(R_{x}\)的系统误差。实验时\(R_{s}\)常用十进位转盘直流电阻箱,其仪器允差为\(\frac{\Delta R_{s}}{R_{s}}=\pm(a + b\frac{m}{R_{s}})\%\),一般常用的 0.1 级十进位转盘电阻箱有\(a = 0.1\),\(b = 0.2\),即\(\Delta R_{s}=(0.001R_{s}+0.002m)\), m=6。
    
    \subsubsection*{电桥灵敏度:}
    为定量确定电桥灵敏度,引入电桥灵敏度概念,定义为\(S=\frac{\Delta d}{\frac{\Delta R_{s}}{R_{s}}}\)。显然,电桥灵敏度S越大,对电桥平衡的判断就越容易,测量结果也更准确,实验中可据此测出所用电桥的灵敏度。在实验中由于电桥灵敏度引入的不确定度\(\Delta_{S}R_{x}\)可用下述方法估测:当电桥达到平衡时略微改变\(R_{s}\),使检流计指针偏离零点 0.2 小格(人眼能察觉的界限),此时可求得\(\Delta_{S}R_{x}=\frac{0.2R_{s}}{S}\),则最终相对不确定度的计算公式应为\(E=\frac{\Delta R_{x}}{R_{x}}=\sqrt{(0.001+\frac{0.002m}{R_{s}})^{2}+(\frac{0.2}{S})^{2}}\)
    
    
    \section{实验重点}
    
    \xiaosi(简述本实验的学习重点,不超过100字,3分)

    \myspace{1}

    掌握惠斯通电桥平衡条件及推导过程;学会搭建实验电路,正确连接电源、灵敏电流计和电阻;能通过调节标准电阻使电桥平衡,进而计算待测电阻阻值。
    
    
    \section{实验难点}
    
    \xiaosi(简述本实验的实现难点,不超过100字,2分)

    \myspace{1}

    灵敏电流计灵敏度高,调节标准电阻时指针偏转微小,难精准判断平衡状态;电路接线多,易出现接线错误导致电桥无法正常工作,需反复检查。



\newpage

\sihao\textbf{注意事项:}
\xiaosi
\begin{enumerate}
    \item  用 PDF 格式上传“实验报告”,文件名:学生姓名+学号+实验名称+周次。
    \item  “实验报告”必须递交在“学在浙大”的本课程的对应实验项目的“作业”模块内。
    \item  “实验报告”成绩必须在“浙江大学物理实验教学中心网站”-“选课系统”内查询。
    \item 教学评价必须在“浙江大学物理实验教学中心网站”-“选课系统”内进行,学生必须进行教学评价,才能看到实验报告成绩,教学评价必须在本次实验结束后 3 天内进行。
\end{enumerate}

\vspace{1cm}
\xiaosi\centering \textbf{浙江大学物理实验教学中心制}

\end{document}