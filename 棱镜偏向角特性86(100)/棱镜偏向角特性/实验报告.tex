\documentclass[10pt,a4paper]{ctexart}
\usepackage[margin=2cm]{geometry}
\usepackage{graphicx}
\usepackage{hyperref}
\usepackage{amsmath}    % 提供方程环境
\usepackage{amssymb} 
\usepackage{physics}    % 简化向量、微分算子语法
\newcommand{\myspace}[1]{\par\vspace{#1\baselineskip}} %自定义空一行
\usepackage{fancyhdr}
\usepackage{wrapfig}
\usepackage{multirow}
\usepackage{booktabs} % 添加 booktabs 宏包
\usepackage{float}
\pagestyle{fancy}
\fancyhf{}  % 清除所有页眉页脚
\fancyfoot[C]{\thepage}  % 在页脚中央设置页码
\renewcommand{\headrulewidth}{0pt}  % 去掉页眉横线
\renewcommand{\footrulewidth}{0pt}  % 去掉页脚横线
% hyperref 已在上方加载一次,避免重复加载
\hypersetup{
    colorlinks=true,
    linkcolor=blue,
    filecolor=magenta,      
    urlcolor=cyan,
    pdftitle={Overleaf Example},
    pdfpagemode=FullScreen,
    }

\usepackage{titlesec}

% 重新定义section格式
\titleformat{\section}
  {\normalfont\Large\bfseries}
  {\chinese{section}、}
  {0pt}
  {}

% 给出常用字号的自定义指令
\newcommand{\chuhao}{\fontsize{42pt}{48pt}\selectfont}
\newcommand{\xiaochu}{\fontsize{36pt}{42pt}\selectfont}
\newcommand{\xiaoer}{\fontsize{18pt}{21.6pt}\selectfont}
\newcommand{\sanhao}{\fontsize{16pt}{20pt}\selectfont}
\newcommand{\xiaosan}{\fontsize{15pt}{18pt}\selectfont}
\newcommand{\sihao}{\fontsize{14pt}{18pt}\selectfont}
\newcommand{\xiaosi}{\fontsize{12pt}{16pt}\selectfont}
\newcommand{\wuhao}{\fontsize{ 10.5pt}{12.6pt}\selectfont}

% 定义中文数字转换
\titleformat{\section}
  {\normalfont\Large\bfseries}
  {\zhnum{section}、}  % 使用\zhnum而不是\chinese
  {0pt}
  {}

% 标题设置
\title{\xiaochu\textbf{物理实验报告}}
\author{}
\date{}

\begin{document}

\vspace*{36pt} % 顶部空白

% 居中填写区域
\begin{center}
    \includegraphics[width = 9cm]{figures/head.png}
    
    \xiaochu\textbf{物理实验报告}
    % 预留空白区域
    \vspace{13pt}
    \vspace{13pt}
    \vspace{13pt}
    \vspace{13pt}
    
    \sanhao\textbf{实验名称:\underline{\makebox[8cm]{ 棱镜偏向角特性 }}} \\[10pt]
    \sanhao\textbf{实验桌号:\underline{\makebox[8cm]{  }}} \\[10pt]
    \sanhao\textbf{指导教师:\underline{\makebox[8cm]{  }}} \\[20pt]
    
    % 预留空白区域
    \myspace{7}
    
   \sihao\textbf{班级:\underline{\makebox[6cm]{cc98}}} \\[10pt]
    \sihao\textbf{姓名:\underline{\makebox[6cm]{Hydrofoil}}} \\[10pt]
    \sihao\textbf{学号:\underline{\makebox[6cm]{324010}}} \\[30pt]
    
    \sihao\textbf{实验日期: \underline{\makebox[0.8cm]{2025}}年 \underline{\makebox[0.4cm]{11}}月\underline{\makebox[0.4cm]{12}}日  星期\underline{\makebox[0.6cm]{三}}上午}
    
    \vspace{18pt}
    \sihao 浙江大学物理实验教学中心
    
\end{center}

\newpage

\section{预习报告}


    \subsection{实验综述}  

    \xiaosi (自述实验现象、实验原理和实验方法,不超过500字,5分)


    \begin{wrapfigure}{r}{0.4\textwidth}
    \centering
    \includegraphics[width=8cm]{figures/棱镜偏向角.png}
    \caption{最小偏向角原理}
    \label{fig:principle}
    \end{wrapfigure}

    \

    \subsubsection*{三棱镜顶角测量原理:}
    三棱镜测量原理在“分光计调整与使用”实验中学习,\(\angle A = \frac{|t_{\text{左}I}-t_{\text{右}I}|+|t_{\text{左}II}-t_{\text{右}II}|}{4}\)

    \subsubsection*{自最小偏向角原理:}

    如图所示,旋转载物台,使光学面 AC 与平行光管射来的光垂直,平行光管发出的平行光射向三棱镜的光学面 AB,经三棱镜光学面 AC 折射出来,望远镜从毛玻璃面 BC 底边出发,逆时针方向旋转,就会看到清晰的汞色单色系列光,说明已经找到折射光路。
    此时再转动载物台,观察该汞单色偏向角的变化,如果向左移动,偏向角$\delta $变小,继续慢慢转动载物台直到汞单色光到达某一位置时突然向左转动,使偏向角$\delta $变大,此转折点即为该单色光的最小偏向角位置。把望远镜对准这个转折位置,并记录此时分光计两游标的读数为\(\theta_{minI}\)和\(\theta_{minII}\),然后移去三棱镜,使望远镜对准入射光(平行光管位置),读取两游标的读数为\(\theta_{0I}\)和\(\theta_{0II}\),则最小偏向角为\(\delta_{min}=\frac{1}{2}(|\theta_{minI}-\theta_{0I}| + |\theta_{minII}-\theta_{0II}|)\)。
    
    \begin{wrapfigure}{r}{0.4\textwidth}
    \centering
    \includegraphics[width=8cm]{figures/折射率测量.png}
    \caption{折射率测量光路图}
    \label{fig:principle}
    \end{wrapfigure}

    \

    \subsubsection*{折射率测量原理:}

    一平行的单色光从三棱镜的一个光学面 AB 入射,经折射后从另一个光学面 AC 射出,如右图所示。入射光和 AB 面法线的夹角为入射角$i$,出射光和 AC 面法线的夹角为出射角\(i'\),入射光和出射光的夹角\(\delta\)就是偏向角。由几何关系知,\(\delta=(i - r)+(i' - r')\),当$i = i'$时,由折射定律有$r = r' $,得\(\delta_{min}=2(i - r)\)。
    
    又因为\(r + r' = 2r = \pi -(\pi - \angle A) = \angle A\)

    所以
    
    \begin{align*}
        i=\frac{A + \delta_{min}}{2}, n=\frac{sin i}{sin r}=\frac{sin \frac{A + \delta_{min}}{2}}{sin \frac{A}{2}}
    \end{align*}
    
    
    因此,只要测出三棱镜顶角A和汞灯单色光入射的最小偏向角\(\delta_{min}\),就可计算出三棱镜对该单色入射光的折射率。
    
    

    \subsubsection*{分光计的调整:}

    按照 “分光计的调整与使用” 实验中的操作步骤,完成对分光计的调整。

    
    \subsubsection*{反射法测量三棱镜顶角:}
    
    按照 “分光计的调整与使用” 实验中的操作步骤,使用反射法完成对三棱镜顶角的测量。

    \subsubsection*{测定三棱镜对汞灯单色光 $\lambda = 546.0nm $(绿光)的最小偏向角}
    
    按要求放置好三棱镜,转动载物台,改变入射角,获得最小偏向角,
    记录分光计两游标的读数为\(\theta_{minI}\)和\(\theta_{minII}\),然后移去三棱镜,望远镜对准入射光(平行光管位置),
    读取游标的读数为\(\theta_{0I}\)和\(\theta_{0II}\),代入\(\delta_{min}=\frac{1}{2}(|\theta_{minI}-\theta_{0I}| + |\theta_{minII}-\theta_{0II}|)\)计算出最小偏向角。

    \subsubsection*{计算三棱镜对各单色光的折射率以及绘制色散曲线:}

    分别测量各单色光的最小偏向角,利用已经测出的三棱镜顶角值,即可由\(n=\frac{sin i}{sin r} =\frac{sin \frac{A + \delta_{min}}{2}}{sin \frac{A}{2}}\)计算出棱镜对各单色光的折射率,制作$n-\lambda $关系曲线(色散曲线)。

    \subsection{实验重点}

    \xiaosi(简述本实验的学习重点,不超过100字,3分)


    \begin{enumerate}   
        \item 进一步熟悉分光计的调整方法;
        \item 测量三棱镜顶角,观察汞灯色散现象;
        \item 掌握最小偏向角的测量方法;
        \item 测定棱镜玻璃对汞灯各单色光的折射率。

    \end{enumerate}  
    
    
    \subsection{实验难点}
    
    \xiaosi(简述本实验的实现难点,不超过100字,2分)

    \begin{enumerate}    
        \item 分光计调平要求高,载物台难完全水平,影响光路精度;
        \item 最小偏向角转折点光线移动缓,主观判断易有偏差;
        \item 刻度盘读数时,游标与刻度线对齐的判断存在主观性误差;
        \item 三棱镜的顶角应接近平台中心偏上的位置,否则难观察反射光,影响顶角测量。
    \end{enumerate}  


\section{原始数据}

    \xiaosi (将有老师签名的“自备数据记录草稿纸”的扫描或手机拍摄图粘贴在下方,20分)

    \begin{figure}[htbp]
        \centering
        \includegraphics[width=16cm]{figures/实验数据.jpg}
        \caption{实验数据}
        \label{fig:图2}
    \end{figure}

    
\section{结果与分析}

    \subsection{数据处理与结果}
    \xiaosi (列出数据表格、选择数据处理方法、给定测量或计算结果,30分)
    
    \

        % 注意:编译此表格需要您的文档已包含 \usepackage{ctex} (或 \usepackage[UTF8]{ctex}) 和 \usepackage{amsmath}
        \textbf{对于绿光偏向角数据处理:}

        \begin{align*}
        U_{A\text{绿}}&=\sqrt{\frac{1}{6(6 - 1)}\sum_{i = 1}^{n}(\delta_{min_i}-\overline{\delta_{min}})^2}=0.01608^{\circ} \\
        U_{B\text{绿}}&=\frac{1}{\sqrt{3}}\times\frac{1'}{60} = 0.00962^{\circ} \\
        \therefore  U_{\text{绿}}&=\sqrt{U_{A\text{绿}}^2 + U_{B\text{绿}}^2}=0.018735^{\circ}\approx1'7'' \\
        \therefore \delta_{\text{绿}}&=51^{\circ}33'50''\pm1'7'' 
        \end{align*}

        \textbf{对于绿光折射率数据处理:}

        \begin{center}
            各次测量下的绿光折射率由公式 $n=\frac{sin \frac{A + \delta_{min}}{2}}{sin \frac{A}{2}}$ 计算得出
        \end{center}
        \begin{align*}
        \overline{n_{\text{绿}}}&= \frac{1}{6} \sum_{i=1}^{6} n_i = 1.65543 \\
        u_{n}&=\sqrt{\frac{1}{6(6 - 1)}\sum_{i = 1}^{n}(n_i-\overline{n})^2}=4.72\times10^{-5} \\
        \therefore n_{\text{绿}} &= 1.65543 \pm5\times10^{-5}
        \end{align*}

        \textbf{列出不同单色光与$\delta $与$\lambda$随$\lambda$变化关系表格:}

        \begin{table}[htbp]
            \centering
            \label{tab:bridge_custom}
            \begin{tabular}{cccc}
                \toprule
                波长 /nm & $\angle A$ & $\delta_{min}$ & 折射率n\\
                \midrule
                577.1(黄光) & $60^\circ$ & \(51^{\circ}16'30''\) & 1.6520 \\
                546.0(绿光) & $60^\circ$ & \(51^{\circ}33'50''\) & 1.6544\\
                435.8(蓝光) & $60^\circ$ & \(53^{\circ}39'30''\) & 1.6734 \\
                404.7(紫光) & $60^\circ$ & \(54^{\circ}46'30''\) & 1.6846\\
                \bottomrule
            \end{tabular}
        \end{table}

        \vspace{5cm}

        \textbf{绘制色散曲线($n-\lambda $曲线),并用色散经验公式拟合得到下图:}

        \begin{figure}[H]
        \centering
        \includegraphics[width=15cm]{figures/nonlinear_fit_chart.jpg}
        \caption{拟合$n-\lambda $曲线}
        \label{fig:图3}
        \end{figure}  

    \subsection{误差分析}
    \xiaosi (运用测量误差、相对误差、不确定度等分析实验结果,20分)
    
    \begin{enumerate}
        \item 载物台难以完全调平,使光路与理论光路产生了一定偏差,对最终结果带来影响;
        \item 在寻找最小偏向角的过程中,光线在转折点附近移动很缓慢,人的主观性对最小偏向角位置的选取存在影响;
        \item 在刻度盘上读数时,判断游标与刻度线对齐时存在主观性,产生读数误差。
    \end{enumerate}

    以上误差总体较小,实验结果中,计算得出的不确定度仅为\(1'7''\),本实验结果较为精确。
    
    
    \subsection{实验探讨}
    \xiaosi (对实验内容、现象和过程的小结,不超过100字,10分)

    \myspace{1}

    通过本次实验,我熟练掌握了分光计调整、三棱镜顶角与最小偏向角测量及折射率计算方法。实验中,精准调平分光计、找准最小偏向角转折点是关键,需耐心细致操作。这让我体会到,物理实验的精度依赖规范操作与严谨态度,主观判断易引入误差,需多次测量减小偏差。同时,色散曲线的绘制让我直观理解了折射率与波长的关系,深化了对光学原理的认知,也提升了数据处理与分析能力。

        
    \section{思考题}

    \xiaosi (解答教材或讲义或老师布置的思考题,10分)

    \subsubsection*{光线经三棱镜折射时,出射角\(i'\)与入射角i满足什么关系时,偏向角\(\delta\)有最小值,写出证明过程。}

    证明:

    假设三棱镜截面是等边三角形,由斯聂耳定律:
    \begin{center}
        \(n sin r_1 = sin i_1 ,n sin r_2 = sin i_2 \)
    \end{center}

    由几何关系, 
    \begin{center}
        $A = r_1 + r_2 $ \\
        $\delta = i_1 + i_2 - A = arcsin(n sin r_1) + arcsin(n sin(A - r_1)) - A $
    \end{center}
    
    对\(r_1\)求导:

    \begin{align*}
    \frac{d\delta}{dr_1}&=\frac{n cos r_1}{\sqrt{1 - n^2 sin^2 r_1}}+\frac{-n cos(A - r_1)}{\sqrt{1 - n^2 sin^2(A - r_1)}} \\
    &=\sqrt{\frac{n^2 - n^2 sin^2 r_1}{1 - n^2 sin^2 r_1}}-\sqrt{\frac{n^2 - n^2 sin^2(A - r_1)}{1 - n^2 sin^2(A - r_1)}} \\
    &\triangleq f(r_1)-f(A - r_1)
    \end{align*}

    由函数性质可知,\(f(r)\)单调递增,所以\(\triangleq f(r_1)-f(A - r_1)\)单调递增。
    
    当\(\frac{d\delta}{dr_1}=0\)时,\(f(r_1)=f(A - r_1)\),即\(r_1 = A - r_1\),故\(r_1 = r_2=\frac{A}{2}\)。
    
    此时\(i_1 = i_2 = arcsin(n sin\frac{A}{2})\),偏向角\(\delta_{min}=2arcsin(n sin\frac{A}{2})\),即偏向角取得最小值。
    
    证毕。
    

\newpage

\sihao\textbf{注意事项:}
\xiaosi
\begin{enumerate}
    \item  用 PDF 格式上传“实验报告”,文件名:学生姓名+学号+实验名称+周次。
    \item  “实验报告”必须递交在“学在浙大”的本课程的对应实验项目的“作业”模块内。
    \item  “实验报告”成绩必须在“浙江大学物理实验教学中心网站”-“选课系统”内查询。
    \item 教学评价必须在“浙江大学物理实验教学中心网站”-“选课系统”内进行,学生必须进行教学评价,才能看到实验报告成绩,教学评价必须在本次实验结束后 3 天内进行。
\end{enumerate}

\vspace{1cm}
\xiaosi\centering \textbf{浙江大学物理实验教学中心制}

\end{document}