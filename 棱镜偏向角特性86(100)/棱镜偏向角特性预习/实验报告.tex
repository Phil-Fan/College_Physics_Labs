\documentclass[10pt,a4paper]{ctexart}
\usepackage[margin=2cm]{geometry}
\usepackage{graphicx}
\usepackage{hyperref}
\usepackage{amsmath}    % 提供方程环境
\usepackage{physics}    % 简化向量、微分算子语法
\newcommand{\myspace}[1]{\par\vspace{#1\baselineskip}} %自定义空一行
\usepackage{fancyhdr}
\usepackage{wrapfig}
\usepackage{booktabs} % 添加 booktabs 宏包
\pagestyle{fancy}
\fancyhf{}  % 清除所有页眉页脚
\fancyfoot[C]{\thepage}  % 在页脚中央设置页码
\renewcommand{\headrulewidth}{0pt}  % 去掉页眉横线
\renewcommand{\footrulewidth}{0pt}  % 去掉页脚横线

\usepackage{hyperref}
\hypersetup{
    colorlinks=true,
    linkcolor=blue,
    filecolor=magenta,      
    urlcolor=cyan,
    pdftitle={Overleaf Example},
    pdfpagemode=FullScreen,
    }

\usepackage{titlesec}

% 重新定义section格式
\titleformat{\section}
  {\normalfont\Large\bfseries}
  {\chinese{section}、}
  {0pt}
  {}

% 给出常用字号的自定义指令
\newcommand{\chuhao}{\fontsize{42pt}{48pt}\selectfont}
\newcommand{\xiaochu}{\fontsize{36pt}{42pt}\selectfont}
\newcommand{\xiaoer}{\fontsize{18pt}{21.6pt}\selectfont}
\newcommand{\sanhao}{\fontsize{16pt}{20pt}\selectfont}
\newcommand{\xiaosan}{\fontsize{15pt}{18pt}\selectfont}
\newcommand{\sihao}{\fontsize{14pt}{18pt}\selectfont}
\newcommand{\xiaosi}{\fontsize{12pt}{16pt}\selectfont}
\newcommand{\wuhao}{\fontsize{10.5pt}{12.6pt}\selectfont}

% 定义中文数字转换
\titleformat{\section}
  {\normalfont\Large\bfseries}
  {\zhnum{section}、}  % 使用\zhnum而不是\chinese
  {0pt}
  {}

% 标题设置
\title{\xiaochu\textbf{物理实验预习报告}}
\author{}
\date{}

\begin{document}

\vspace*{36pt} % 顶部空白

% 居中填写区域
\begin{center}
    \includegraphics[width = 9cm]{head.png}
    
    \xiaochu\textbf{物理实验预习报告}
    % 预留空白区域
    \vspace{13pt}
    \vspace{13pt}
    \vspace{13pt}
    \vspace{13pt}
    
    \sanhao\textbf{实验名称:\underline{\makebox[8cm]{ 棱镜偏向角特性 }}} \\[10pt]
    \sanhao\textbf{实验桌号:\underline{\makebox[8cm]{  }}} \\[10pt]
    \sanhao\textbf{指导教师:\underline{\makebox[8cm]{  }}} \\[20pt]
    
    % 预留空白区域
    \myspace{7}
    
   \sihao\textbf{班级:\underline{\makebox[6cm]{cc98}}} \\[10pt]
    \sihao\textbf{姓名:\underline{\makebox[6cm]{Hydrofoil}}} \\[10pt]
    \sihao\textbf{学号:\underline{\makebox[6cm]{324010}}} \\[30pt]
    
   \sihao\textbf{实验日期: \underline{\makebox[0.8cm] 2025 }年 \underline{\makebox[0.4cm]11}月\underline{\makebox[0.4cm]12}日  星期\underline{\makebox[0.6cm]三}上午}
    
    \vspace{18pt}
    \sihao 浙江大学物理实验教学中心
    
\end{center}

\newpage


    \section{实验综述}
    
    \xiaosi (自述实验现象、实验原理和实验方法,不超过500字,5分)
    
    \myspace{0}

    
    \begin{wrapfigure}{r}{0.4\textwidth}
    \centering
    \includegraphics[width=7cm]{棱镜偏向角.png}
    \caption{最小偏向角原理}
    \label{fig:principle}
    \end{wrapfigure}
    
    \
    
    \subsubsection*{三棱镜顶角测量原理:}
    三棱镜测量原理在“分光计调整与使用”实验中学习,\(\angle A = \frac{|t_{\text{左}I}-t_{\text{右}I}|+|t_{\text{左}II}-t_{\text{右}II}|}{4}\)

    \subsubsection*{自最小偏向角原理:}

    如图所示,旋转载物台,使光学面 AC 与平行光管射来的光垂直,平行光管发出的平行光射向三棱镜的光学面 AB,经三棱镜光学面 AC 折射出来,望远镜从毛玻璃面 BC 底边出发,逆时针方向旋转,就会看到清晰的汞色单色系列光,说明已经找到折射光路。
    此时再转动载物台,观察该汞单色偏向角的变化,如果向左移动,偏向角$\delta $变小,继续慢慢转动载物台直到汞单色光到达某一位置时突然向左转动,使偏向角$\delta $变大,此转折点即为该单色光的最小偏向角位置。把望远镜对准这个转折位置,并记录此时分光计两游标的读数为\(\theta_{minI}\)和\(\theta_{minII}\),然后移去三棱镜,使望远镜对准入射光(平行光管位置),读取两游标的读数为\(\theta_{0I}\)和\(\theta_{0II}\),则最小偏向角为\(\delta_{min}=\frac{1}{2}(|\theta_{minI}-\theta_{0I}| + |\theta_{minII}-\theta_{0II}|)\)。
    
    \begin{wrapfigure}{r}{0.4\textwidth}
    \centering
    \includegraphics[width=8cm]{折射率测量.png}
    \caption{折射率测量光路图}
    \label{fig:principle}
    \end{wrapfigure}

    \

    \subsubsection*{折射率测量原理:}

    一平行的单色光从三棱镜的一个光学面 AB 入射,经折射后从另一个光学面 AC 射出,如右图所示。入射光和 AB 面法线的夹角为入射角$i$,出射光和 AC 面法线的夹角为出射角\(i'\),入射光和出射光的夹角\(\delta\)就是偏向角。由几何关系知,\(\delta=(i - r)+(i' - r')\),当$i = i'$时,由折射定律有$r = r' $,得\(\delta_{min}=2(i - r)\)。
    
    又因为\(r + r' = 2r = \pi -(\pi - \angle A) = \angle A\)

    所以
    
    \begin{align*}
        i=\frac{A + \delta_{min}}{2}, n=\frac{sin i}{sin r}=\frac{sin \frac{A + \delta_{min}}{2}}{sin \frac{A}{2}}
    \end{align*}
    
    
    因此,只要测出三棱镜顶角A和汞灯单色光入射的最小偏向角\(\delta_{min}\),就可计算出三棱镜对该单色入射光的折射率。
    
    

    \subsubsection*{分光计的调整:}

    按照 “分光计的调整与使用” 实验中的操作步骤,完成对分光计的调整。

    
    \subsubsection*{反射法测量三棱镜顶角:}
    
    按照 “分光计的调整与使用” 实验中的操作步骤,使用反射法完成对三棱镜顶角的测量。

    \subsubsection*{测定三棱镜对汞灯单色光 $\lambda = 546.0nm $(绿光)的最小偏向角}
    
    按要求放置好三棱镜,转动载物台,改变入射角,获得最小偏向角,
    记录分光计两游标的读数为\(\theta_{minI}\)和\(\theta_{minII}\),然后移去三棱镜,望远镜对准入射光(平行光管位置),
    读取游标的读数为\(\theta_{0I}\)和\(\theta_{0II}\),代入\(\delta_{min}=\frac{1}{2}(|\theta_{minI}-\theta_{0I}| + |\theta_{minII}-\theta_{0II}|)\)计算出最小偏向角。

    \subsubsection*{计算三棱镜对各单色光的折射率以及绘制色散曲线:}

    分别测量各单色光的最小偏向角,利用已经测出的三棱镜顶角值,即可由\(n=\frac{sin i}{sin r} =\frac{sin \frac{A + \delta_{min}}{2}}{sin \frac{A}{2}}\)计算出棱镜对各单色光的折射率,制作$n-\lambda $关系曲线(色散曲线)。

    \section{实验重点}

    \xiaosi(简述本实验的学习重点,不超过100字,3分)


    \begin{enumerate}   
        \item 进一步熟悉分光计的调整方法;
        \item 测量三棱镜顶角,观察汞灯色散现象;
        \item 掌握最小偏向角的测量方法;
        \item 测定棱镜玻璃对汞灯各单色光的折射率。

    \end{enumerate}  

    
    \section{实验难点}
    
    \xiaosi(简述本实验的实现难点,不超过100字,2分)

    \begin{enumerate}
        \item 实验中电阻温度变化快,读取数据应在同一时刻,否则会造成较大误差;
        \item 应加热装置 PID 调节需反复微调,响应滞后,易超调或不达设定温,影响数据稳定性,难满足实验精度。
    \end{enumerate}



\newpage

\sihao\textbf{注意事项:}
\xiaosi
\begin{enumerate}
    \item  用 PDF 格式上传“实验报告”,文件名:学生姓名+学号+实验名称+周次。
    \item  “实验报告”必须递交在“学在浙大”的本课程的对应实验项目的“作业”模块内。
    \item  “实验报告”成绩必须在“浙江大学物理实验教学中心网站”-“选课系统”内查询。
    \item 教学评价必须在“浙江大学物理实验教学中心网站”-“选课系统”内进行,学生必须进行教学评价,才能看到实验报告成绩,教学评价必须在本次实验结束后 3 天内进行。
\end{enumerate}

\vspace{1cm}
\xiaosi\centering \textbf{浙江大学物理实验教学中心制}

\end{document}